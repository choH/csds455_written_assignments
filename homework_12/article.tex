\documentclass[11pt]{article}
\usepackage{setspace}
\setstretch{1}
\usepackage{amsmath,amssymb, amsthm}
\usepackage{graphicx}
\usepackage{bm}
\usepackage[hang, flushmargin]{footmisc}
\usepackage[colorlinks=true]{hyperref}
\usepackage[nameinlink]{cleveref}
\usepackage{footnotebackref}
\usepackage{url}
\usepackage{listings}
\usepackage[most]{tcolorbox}
\usepackage{inconsolata}
\usepackage[papersize={8.5in,11in}, margin=1in]{geometry}
\usepackage{float}
\usepackage{caption}
\usepackage{esint}
\usepackage{url}
\usepackage{enumitem}
\usepackage{subfig}
\usepackage{wasysym}
\newcommand{\ilc}{\texttt}
\usepackage{etoolbox}
\usepackage{algorithm}
\usepackage{changepage}
% \usepackage{algorithmic}
\usepackage[noend]{algpseudocode}
\usepackage{tikz}
\usetikzlibrary{matrix,positioning,arrows.meta,arrows}
\patchcmd{\thebibliography}{\section*{\refname}}{}{}{}
% \PassOptionsToPackage{hyphens}{url}\usepackage{hyperref}

\providecommand{\myceil}[1]{\left \lceil #1 \right \rceil }
\providecommand{\myfloor}[1]{\left \lfloor #1 \right \rfloor }


\begin{document}



\title{\textbf{CSDS 455: Homework 12}}

\author{Shaochen (Henry) ZHONG, \ilc{sxz517}}
\date{Due and submitted on 10/05/2020 \\ Fall 2020, Dr. Connamacher}
\maketitle


\section*{Problem 1}

Assume we have graph $G'$ which is identical with $G$, but already colored by $\chi(G)$ colors. We may have a vertex ording of $v_1, v_2, ..., v_n$ (assume $|V(G)| = n$), where $v_1$ to $v_i$ have color $c_1$, $v_{i+1}$ to $v_{j}$ have color $c_2$, ..., and $v_k$ to $v_n$ have color $c_{\chi(G)}$ (for $i < j < k < n$).

This votex odering will guarteen a $\chi(G)$-coloring of $G$. As if the a vertex $v$ has a color of $c$ in $G'$, then the next vertex $v'$ from the above ordering will either have the same color as $v$ in $G'$, in this case the greedy algorithm will color $v'$ the same color as $v$; or it will have a different color, in this case the greedy algorithm will try all ``used'' colors\footnote{Colors used on vertices before $v'$ in the listed ordering} -- which obviously not going to work as $G'$ represents the minimum colored graph of $G$ -- then the algorithm will give $v'$ a new color. Following this order the algorithm will color $G$ just as $G'$, which contains $\chi(G)$ colors.


\section*{Problem 2}
\section*{Problem 3}
\section*{Problem 4}
% \section{References}
%
% \nocite{*}
% \raggedright
% \bibliography{references.bib}
% \bibliographystyle{plain}


\end{document}