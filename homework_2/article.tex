\documentclass[11pt]{article}
\usepackage{setspace}
\setstretch{1}
\usepackage{amsmath,amssymb, amsthm}
\usepackage{graphicx}
\usepackage{bm}
\usepackage[hang, flushmargin]{footmisc}
\usepackage[colorlinks=true]{hyperref}
\usepackage[nameinlink]{cleveref}
\usepackage{footnotebackref}
\usepackage{url}
\usepackage{listings}
\usepackage[most]{tcolorbox}
\usepackage{inconsolata}
\usepackage[papersize={8.5in,11in}, margin=1in]{geometry}
\usepackage{float}
\usepackage{caption}
\usepackage{esint}
\usepackage{url}
\usepackage{enumitem}
\usepackage{subfig}
\usepackage{wasysym}
\newcommand{\ilc}{\texttt}
\usepackage{etoolbox}
\usepackage{algorithm}
% \usepackage{algorithmic}
\usepackage[noend]{algpseudocode}
\usepackage{tikz}
\usetikzlibrary{matrix,positioning,arrows.meta,arrows}
\patchcmd{\thebibliography}{\section*{\refname}}{}{}{}


\begin{document}



\title{\textbf{CSDS 455: Homework 2}}

\author{Shaochen (Henry) ZHONG, \ilc{sxz517}}
\date{Due and submitted on 08/31/2020 \\ CSDS 455, Dr. Connamacher}
\maketitle

\section{Problem 1}

If the algorithm perform the \ilc{BFS} in the following way, it is possible to achieve the demonstrated \textit{maximal matching} during middle of the iteration process.

For the ease of description, let's label the nodes from left set ($U$) of the bipartite graph $A, B, C, D, E, F$ from top to bottom; and label the nodes from right set ($V$) of the the bipartite graph as $(1), (2), (3), (4), (5)$ from top to bottom. And we have a empty list $M$.

\begin{itemize}
    \item Find a unchecked node $\in U$ labelled with biggest alphabetical value, let the node be $\alpha$
    \item Gather all child nodes of $\alpha$ $\in V$ into a list $\beta$, then sort $\beta$ decendingly based on each node's degree (number of edges connected). If two nodes have the same degree, the node with lower numerical value will store prior in the list.
    \item Pop the first node from $\beta$ to check until $\beta$ is empty.
    \item Find a unchecked node $\in U$ labelled with biggest alphabetical value and repeat the whole list.
\end{itemize}

Here's a emulation of a \textsc{Hopcroft-Karp} algorithm implemented with the above \ilc{BFS}.

\begin{itemize}
    \item The unchecked node with largest alphabetical value $\in U$ is node $F$.
    \item The child nodes of $F$ $\in V$ are $\{(1), (4)\}$. With $\delta{(1)} = 1$ and $\delta{(4)} = 3$. Therefore edge $<F, (4)>$ is checked. Since it is a single edge argumenting path, it is added to $M$.
    \item The unchecked node with largest alphabetical value $\in U$ is node $E$.
    \item The child nodes of $E$ $\in V$ are $\{(3), (5)\}$. With $\delta{(3)} = 3$ and $\delta{(4)} = 5$. Since node $(3)$ has a lower numerical value, edge $<E, (3)>$ is checked. Since it is a single edge argumenting path, it is added to $M$.
    \item The unchecked node with largest alphabetical value $\in U$ is node $D$.
    \item The child nodes of $D$ $\in V$ are $\{(2), (5)\}$. With $\delta{(2)} = 1$ and $\delta{(5)} = 3$. Therefore edge $<D, (5)>$ is checked. Since it is a single edge argumenting path, it is added to $M$.
\end{itemize}

The given \textit{maximal matching} is obtained through the above iteration of \textsc{Hopcroft-Karp} algorithm. The statement is proven by direct example.

\section{Problem 2}

This problem will inherit the labeling and list $M$ introduced by the above problem.

For $C, B, A \in U$, all of their child nodes $\in V$ are not free. Thus, we will check for the free nodes $\in U$ for non-single-edged \textit{argumenting paths}.\newline

Starting with node $C$, as it has the largest alphabetical value of all free nodes $\in U$. It has two child nodes $\in V$, $\{(4), (5)\}$, both having a degree of $3$. Thus we first check node $(4)$ since it has a lower numerical value. Since we are looking for an \textit{argumenting paths}, the only node we can proceed from $(4)$ is $F$, as it is the only \textit{matching edge} connected to $(4)$. The only node we can proceed from $F$ is $(1)$, as it is the only free node conncted to $F$.

Now, with DFS from $(1)$, we have discovered an \textit{argumenting paths} of $<C, (4), F, (1)>$. Thus, we remove the \textit{matching edge} $<(4), F>$ from $M$, then add edge $<C, (4)>$ and $<F, (1)>$ to $M$.\newline

Since there will be no more \textit{argumenting paths} in the graph (as we can't find any more alternating path which starts and ends on free nodes). This $\{ <F, (1)>, <E, (3)>, <D, (5)>, <C, (4)> \}$, with a cardinality of $4$, be the \textit{maximum matching} for the graph.

\section{Problem 3}


% \section{References}
%
% \nocite{*}
% \raggedright
% \bibliography{references.bib}
% \bibliographystyle{plain}


\end{document}