\documentclass[11pt]{article}
\usepackage{setspace}
\setstretch{1}
\usepackage{amsmath,amssymb, amsthm}
\usepackage{graphicx}
\usepackage{bm}
\usepackage[hang, flushmargin]{footmisc}
\usepackage[colorlinks=true]{hyperref}
\usepackage[nameinlink]{cleveref}
\usepackage{footnotebackref}
\usepackage{url}
\usepackage{listings}
\usepackage[most]{tcolorbox}
\usepackage{inconsolata}
\usepackage[papersize={8.5in,11in}, margin=1in]{geometry}
\usepackage{float}
\usepackage{caption}
\usepackage{esint}
\usepackage{url}
\usepackage{enumitem}
\usepackage{subfig}
\usepackage{wasysym}
\newcommand{\ilc}{\texttt}
\usepackage{etoolbox}
\usepackage{algorithm}
\usepackage{changepage}
% \usepackage{algorithmic}
\usepackage[noend]{algpseudocode}
\usepackage{tikz}
\usetikzlibrary{matrix,positioning,arrows.meta,arrows}
\patchcmd{\thebibliography}{\section*{\refname}}{}{}{}
% \PassOptionsToPackage{hyphens}{url}\usepackage{hyperref}

\providecommand{\myceil}[1]{\left \lceil #1 \right \rceil }
\providecommand{\myfloor}[1]{\left \lfloor #1 \right \rfloor }


\begin{document}



\title{\textbf{CSDS 455: Homework 5}}

\author{Shaochen (Henry) ZHONG, \ilc{sxz517}}
\date{Due and submitted on 09/09/2020 \\ CSDS 455, Dr. Connamacher}
\maketitle

\section{Problem 1}

\textit{For this quetion I have consulted \url{http://ion.uwinnipeg.ca/~ychen2/advancedAD/notes-March15.pdf} and \url{https://math.la.asu.edu/~andrzej/teach/mat416/proofs3.pdf}.}\newline


We assume $G$ to be the maximal counterexample as it has no 1-factor, but having $G' = G+e$ will have a 1-factor. Since $q(G'-S) \leq q(G-S)$ (we know this from \textit{Class 5 Pratice: Q1}), the Tutte's condition is still satisfied. We want to show a contradiction of if $G'$ can have a 1-factor, so does $G$.\newline

Let $U$ be the set of vertices in $G$ with degree $|V(G)-1$ (every vertex if $V$ connected to every other vertices in $G$). If $G-U$ are consist of disjoint complete graphs, then $q(G-U) \leq |U|$ (because a component is formed by removing a vertex $\in U$ from the connection). We can find a perfect matching out of it by finding 1-factor from each even component, and connect $1$ unmatched vertex in each odd component to one vertext $U$, then connect the unmacthed vertices in $U$ to each other (we know this from \textit{Class 5 Pratice: Q2}).\newline

If $G-U$ are not disjoint union of complete graphs, there must be vertices $u, v$ which are in the same component of $G-U$, both conncted to a vertex $w$, but not adjacent to each other. This means we have $uw, uw \in E(G)$, but not $uv$. Now we locate a vertex $z$ under the same component of $\in G-U$ and there is no edge $wz$, there must be one as otherwise $z$ will be in $U$.

Since we asssume $G'$ has a one factor, denotes a 1-factor $M_1$ for $G+uv$, anda 1-factor $M_2$ for $G+wz$. Let $F = M_1 \Delta M_2$, we must have $uv, wz \in F$ as they are only in $M_1$ or $M_2$, never both. This suggests $F$ is consisted by every even cycle of $G$ which traversed through all $V(G)$.\newline

Let even cycle $C_1 \in F$ contains $uv$ but not $wz$, and even cycle $C_2 \in F$ contains $wz$ but not $uv$. Due to the even cycle nature, we can always find a 1-factor of $C_1$ without taking $uv$ as a match (say the 1-factor of $C_1$ with $uv$ being a matched edge is $M_{uv}$, the alternative 1-factor would be $C_1 - M_{uv}$); same goes to $C_2$ by not taking $wz$ as a match. Since we can do this to every $C \in F$, this suggests if $G'$ may have a 1-factor with an extra edge $e$, so does $G$ -- the contradiction is found in this case.

However, it is possible to have an even cycle $C \in F$ which contains both $uv$ and $wz$ at the same time. Known that there is $uw, vw \in E(G)$, let path $P_1 \in C$ to be between $w$ and $u$, and $P_2 \in C$ to be between $w$ and $v$. For $N_1 = E(P_2) \cap M_1$ and $N_2 = E(P_1) \cap M_2$, we have $(N_1 \cup \N_2 \cup \ \{wu\})  \ \cup \ (M_1 - E(C))$ being a 1-factor of $G$. Again, since we can do this to every compunent $C \in F$, we can always find a 1-factor of $G$ in this case.\newline

Since a contradiction can be found in all possible cases, the statement is proven by contradiction.








\section{Problem 2}

\textit{I worked with -- or technically, I learned from -- Yige Sun for this question.}\newline

$\Longrightarrow$: For $G$ containing a $k$-factor, every $A(v) $must be connected to $k$ other vertices in other $A$ partitions. Thus, $d(v) - k$ vertices will have a perfect matching with vertices in $B(v)$. Combine edges of partitions by traverse through different vertices in $G$, we have a perfect matching for $H$, thus $H$ has a 1-factor.\newline

$\Longleftarrow$: For $H$ having a 1-factor, every vertices in every $B(v)$ will matched to $d(v)-k$ vertices in its corresponding $A(v)$. By removing all the matched edges and vertcies, we have $|A(v)| = d(v) - (d(v) - k) = k$ for every $v \in G$. This implies that every vertex in $G$ are at least conncted to $k$ other vertices in $G$. Taking these connections, we will have $k-factor$ in $G$.


% \section{References}
%
% \nocite{*}
% \raggedright
% \bibliography{references.bib}
% \bibliographystyle{plain}


\end{document}