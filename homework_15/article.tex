\documentclass[11pt]{article}
\usepackage{setspace}
\setstretch{1}
\usepackage{amsmath,amssymb, amsthm}
\usepackage{graphicx}
\usepackage{bm}
\usepackage[hang, flushmargin]{footmisc}
\usepackage[colorlinks=true]{hyperref}
\usepackage[nameinlink]{cleveref}
\usepackage{footnotebackref}
\usepackage{url}
\usepackage{listings}
\usepackage[most]{tcolorbox}
\usepackage{inconsolata}
\usepackage[papersize={8.5in,11in}, margin=1in]{geometry}
\usepackage{float}
\usepackage{caption}
\usepackage{esint}
\usepackage{url}
\usepackage{enumitem}
\usepackage{subfig}
\usepackage{wasysym}
\newcommand{\ilc}{\texttt}
\usepackage{etoolbox}
\usepackage{algorithm}
\usepackage{changepage}
% \usepackage{algorithmic}
\usepackage[noend]{algpseudocode}
\usepackage{tikz}
\usetikzlibrary{matrix,positioning,arrows.meta,arrows}
\patchcmd{\thebibliography}{\section*{\refname}}{}{}{}
% \PassOptionsToPackage{hyphens}{url}\usepackage{hyperref}

\providecommand{\myceil}[1]{\left \lceil #1 \right \rceil }
\providecommand{\myfloor}[1]{\left \lfloor #1 \right \rfloor }


\begin{document}



\title{\textbf{CSDS 455: Homework 15}}

\author{Shaochen (Henry) ZHONG, \ilc{sxz517}}
\date{Due and submitted on 10/14/2020 \\ Fall 2020, Dr. Connamacher}
\maketitle

\textit{I have consulted \url{https://scholarworks.wmich.edu/cgi/viewcontent.cgi?article=3802&context=honors_theses} for this assignment.}

\section*{Problem 1}

It is because for the regularity to be hold, we will need the relationship of $|d(X, Y)| - d(A, B)| < \epsilon$. For $e(X, Y)$ representing the number of edges between $X$ and $Y$, the density of pair $(X, Y)$ is determind by $\frac{e(X, Y)}{|X||Y|}$ and will therefore be a value between 0 to 1. This is because $e(X, Y)$ can be at most $|X||Y|$, and at least $0$ as neither $X$ or $Y$ can be an empty set.

For the relationship $|d(X, Y)| - d(A, B)| < \epsilon$ to be hold, tthe $X, Y$ sets we pick can't be too small. An extreme case will be $|X| = |Y| = 1$, where each set has one and only one vertex, then $d(X, Y)$ will be either 0 or 1 depending on if there's an edge between the two vertices. In the case of $d(X, Y) = 0$, we will inevitably have $|d(X, Y)| - d(A, B)| = d(A, B)$, which might not be $< \epsilon$. Therefore, a minimum vertex cardinality of $X, Y$ is required to give the subsets similar density to their parent sets.


\section*{Problem 2}

Due to the nature of complement graph, we will have $d_{\bar{G}}(X, Y) = 1 - d_G(X, Y)$ (W.L.O.G). This is because as the edges not connected in $G$ are now connected in $\bar{G}$, for the density formula $d(X, Y) = \frac{e(X, Y)}{|X||Y|}$\footnote{$X, Y$ here are just two example sets} the denominator will be the same, but for the numerator the edges between $\bar{X}$ and $\bar{Y}$ will be $|X||Y| - e(X, Y)$.

Subsituting this discover into the $\epsilon$-regularity inequality and knowing that $|d_G(X, Y)| - d_G(A, B)| < \epsilon$, we have:

\begin{align*}
    |d_{\bar{G}}(X, Y)| - d_{\bar{G}}(A, B)| &= |1 - d_G(X, Y) - (1 - d_G(A, B))| \\
    &= |d_G(A, B) - d_G(X, Y)| \\
    &< \epsilon
\end{align*}

Thus for an $\epsilon$-regular $G$, we have an $\epsilon$-regular $\bar{G}$.



% \section{References}
%
% \nocite{*}
% \raggedright
% \bibliography{references.bib}
% \bibliographystyle{plain}


\end{document}