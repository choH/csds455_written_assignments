\documentclass[12pt]{article}
\usepackage{setspace}
\setstretch{1}
\usepackage{amsmath,amssymb, amsthm}
\usepackage{graphicx}
\usepackage{bm}
\usepackage[hang, flushmargin]{footmisc}
\usepackage[colorlinks=true]{hyperref}
\usepackage[nameinlink]{cleveref}
\usepackage{footnotebackref}
\usepackage{url}
\usepackage{listings}
\usepackage[most]{tcolorbox}
\usepackage{inconsolata}
\usepackage[papersize={8.5in,11in}, margin=1in]{geometry}
\usepackage{float}
\usepackage{caption}
\usepackage{esint}
\usepackage{url}
\usepackage{enumitem}
\usepackage{subfig}
\usepackage{wasysym}
\newcommand{\ilc}{\texttt}
\usepackage{etoolbox}
\usepackage{algorithm}
% \usepackage{algorithmic}
\usepackage[noend]{algpseudocode}
\usepackage{tikz}
\usetikzlibrary{matrix,positioning,arrows.meta,arrows}
\patchcmd{\thebibliography}{\section*{\refname}}{}{}{}


\begin{document}



\title{\textbf{CSDS 455: Homework 1}}

\author{Shaochen (Henry) ZHONG, \ilc{sxz517}}
\date{Due and submitted on 08/26/2020 \\ CSDS 455, Dr. Connamacher}
\maketitle

\section{Problem 1}

\begin{itemize}
    \item \textbf{Proposition 1}: There exists a self-complementary graph $G$ with $n$ vertices .
    \item \textbf{Proposition 2}: For $n \equiv 0$ or $n \equiv 1 \mod 4$.
\end{itemize}

\begin{proof}
\textbf{Proposition 1} $\longrightarrow$ \textbf{Proposition 2}, to prove by direct proof:\newline

Let graph $G$ be a self-complementary graph with $n$ vertices, and let $G'$ be a completed graph based on the vertices of $G$. It is known that $G$ must have half of the edges of $G'$ to be self-complementary. Thus, we may tell $G$ has:

\begin{equation}
    \frac{{n \choose 2}}{2} = \frac{n(n-1)}{4} \nonumber
\end{equation}

number of edges.

Since the number of edges must be $\in \mathbb{Z}$, we must have $n$ or $(n-1)$ being divisible by $4$. Therefore, for a self-complementary graph with $n$ vertices, we must have $n \equiv 0$ or $n \equiv 1 \mod 4$.
\end{proof}

\begin{proof}
\textbf{Proposition 2} $\longrightarrow$ \textbf{Proposition 1}, to prove by induction:\newline

It is known that we have $P_4$ (a line that connects four vertices) and $C_5$ (a pentagon that connects five vertices) being self-complementary graphs, we call these graphs $\alpha$. Now we have another $\beta$ graph of $P_4$, and we connects all the vertices of graph $\alpha$ to the two vertices with a degree of $2$ from graph $\beta$ to form a new graph $\alpha'$.

This new graph $\alpha'$ will still be self-complementary since the $\alpha$ graph is self-complementary, the $\beta$ graph is self-complementary; and the connections between $\alpha$ and $\beta$ will also be self-complementary due to the fact that the two degree-$1$ vertices in $\beta$ that did NOT joined $\alpha$ will be the degree-$2$ vertices in $\bar{\beta}$ that joining $\bar{\alpha}$ with $\bar{\beta}$ -- which is isomorphic to the structure of $\alpha'$ in terms of the connections between $\alpha$ and $\beta$.\newline

Following this process, we can always contruct a self-complementary graph with $n$ vertices with $n \equiv 0 \mod 4$ (building from $P_4$)  or $n \equiv 1 \mod 4$ (building from $C_5$). Thus, for graph $G$ with $n$ vertices where $n \equiv 0$ or $n \equiv 1 \mod 4$, there exists a self-complementary graph.
\section{Problem 2}

\begin{proof}
To prove by contradiction:\newline
Assume there exists two disjoined paths, $P$ and $Q$, both being the longest paths in a connected graph $G$ (with length $L$), where we have vertices $p \in P$ and $q \in Q$. Since $P$ and $Q$ have no shared vertex, also since $G$ is a connected graph, there must be a path $R$ which connects vertex $p$ to $q$, where the length of $R$ must be $\geq 1$.

Make $p$ and $q$ being the mid-point of $P$ and $Q$ respectively, we may have a new path $S$ travels from the first half of $P$ (till vertex $p$), then go through $R$ (to vertex $q$), then travels thought the second half of $Q$. This path $S$ will have a length of $\ilc{1/2 len(P)} + \ilc{len(R)} + \ilc{1/2 len(Q)}$ which is at least $\frac{1}{2}L + 1 + \frac{1}{2}L$, which is $\geq L$. Thus, by contradiction, if $P$ and $Q$ are the longest paths in a connected graph, they must have a common vertex.

\end{proof}


\section{Problem 3}

% \begin{proof}
% \textit{(i)} $\longrightarrow$ \textit{(ii)}, to prove by direct proof:
% \end{proof}



\begin{proof}

\newline Set \textit{(i)} as base. \newline

\textit{(i)} $\longrightarrow$ \textit{(ii)}, to prove by contradiction:

Assume there are two paths in $T$ between two arbitrarily selected nodes $u$ and $v$, then we can form a cycle on nodes $(u, v)$, which is against the definition of a tree. Thus, by contradiction, the statement is proven \newline

\textit{(ii)} $\longrightarrow$ \textit{(i)}, to prove by contrapositive:

Assume $T$ is not a tree, which means it is not a connected acyclic graph, it must has a cycle of same fashion or some nodes of $T$ is not conncted to others. Both of the situations void proposition \textit{(ii)}, as if there is a cycle then the path between a certain two-node are no long unique; and if some nodes are not conncted, then there is no path (not to mention ``unique path'') at all. Thus, by contrapositive, the statement is proven. \newline

\textit{(ii)} $\longrightarrow$ \textit{(iii)}, to prove by contradiction:

Let $e$ to be the edge of $(u, v)$ in $T$. If $T-e$ is disconncted, this means there is another path to connect $u$ with $v$. Which is a contradiction to proposition \textit{(ii)}. Thus, the statement is proven. \newline

\textit{(iii)} $\longrightarrow$ \textit{(ii)}, to prove by contrapositive:

Assume there are two path connecting nodes $u, v$ in $T$ (a negation of proposition \textit{(ii)}), while one of them is a direct edge $(u, v)$, lets set it to be $e$. Then when we delete $e$, we may have another path still connecting $u, v$, which makes the new $T'$ still being conncted (a negation of proposition \textit{(iii)}). Thus, the statement is proven by contrapositive. \newline

\textit{(ii)} $\longrightarrow$ \textit{(iv)}, to prove by direct proof:


For every two nodes in $T$ to be uniquely conncted, $T$ is connected, has no cycle, and any two non-adjacent nodes $u, v$ in $T$ are not directly connected -- as otherwise node $u$ and $v$ can have at aleast two paths between them (the direct connection and indirect connection $<u, ..., v>$ since $T$ is conncted) and this forms a cycle. Let $xy$ to be edge $(u, v)$, a cycle is formed with this edge $(u, v)$ and original path between $u, v$ in $T$. Thus, the statement is proven. \newline


\textit{(iv)} $\longrightarrow$ \textit{(ii)}, to prove by direct proof:

For a $T$ which does not contain any cycle, and we can find a cycle for any two non-adjacent nodes $x, y$ in $T+xy$. We can tell that without the direct edge of $(x,y)$, node $x$ and $y$ are still conncted in $T$. This is because a cycle of $x, y$ needs a minimum of two paths between node $x$ and $y$, so there must be another path from $x$ to $y$ exists in $T$. Since $x$ and $y$ can be any two non-adjacent nodes, and since any two adjacent nodes are intrinsically connected, we may say that all nodes in $T$ are connected. With $T$ being connected and having no cycle, we may say that every two nodes in $T$ are linked by a unique path. Thus, the statement is proven with direct proof.



\end{proof}

\section{Problem 4}


\begin{proof}
To prove by induction: \newline

For $T_1$ being a set of trees with $1$ node (in this case, there is only one tree). Any tree from $T_1$ must be a subgraph of graph $G_1$, where $G_1$ represents all the graphs with $\delta(G_1) \geq 0$ -- as both the tree and the graph is essentially a single node with no edges. \newline

Assume a random tree $T_k$ from set $T_K$, a set of trees with $k$ nodes, such tree must be a subgraph of a graph $G_k$ from set $G_K$, where $\delta(G_k) \geq k - 1$. Such randomly selected $T_k$ will also be a subgraph of a certain graph from set $G_{K+1}$; since $G_{K+1}$, with $\delta(G_(k+1)) \geq k$, is included in set $G_K$ due to $\delta(G_(k+1)) \geq \delta(G_k) \geq k - 1$.\newline

If we locate an arbitary node $p$ from tree $T_{k}$, then add a child node $q$ with a corresponding edge to $p$  (namely, edge $(p, q)$), we have a new tree $T'$. This $T'$ will be a potential $T_{k+1}$ tree from set $T_{K+1}$ -- since with the newly added node $q$, now it has $k+1$ nodes.

Now take graph $G_k$ and add a new node $q'$ that is connected to all nodes of $G_k$, we call this modified version of $G_k$ as $G_k'$. Such graph $G_k'$ is considered to be a graph from set $G_{K+1}$. Since node $q'$ is conncted to every node of $G_k$, thus it increases $\delta(G_k')$ by $1$ to be at least $\delta(G_k) + 1$, namely $\delta(G_k') \geq k$ -- which is the exact requirement for $\delta(G_{K+1})$.\newline

We now may show $T'$ is a subgraph of $G_k'$. We may locate the subgraph of $T_k$ in $G_k'$, and find the projection of node $p$ from $T_k$ in $G_k'$ (let's call it ``$p'$''). This $p'$ node is connected to the $q'$ node we just added to $G_k'$, and the subgraph of $T_k$ in $G_k'$ plus this $p'$ node will form a subgraph that is isomorphic to $T'$ -- since $T'$ is essentially $T_k$ with an extra node $q$ connected to $p$.

Since the node $p$ in $T'$ is an arbitrarily selected node in randomly selected tree from set $T_K$, this $T'$ can essentially be any tree in set $T_{K+1}$. Thus, any tree from set $T_{K+1}$ may have a corresponding graph from set $G_{K+1}$, where the tree will be a subgraph of the graph, namely $T_{k+1} \subseteq G_{k+1}$. Since $T_1$, $T_k$, and $T_{k+1}$ are all true, we have proven the statement by induction.


\end{proof}

% \section{References}
%
% \nocite{*}
% \raggedright
% \bibliography{references.bib}
% \bibliographystyle{plain}


\end{document}