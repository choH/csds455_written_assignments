\documentclass[12pt]{article}
\usepackage{setspace}
\setstretch{1}
\usepackage{amsmath,amssymb, amsthm}
\usepackage{graphicx}
\usepackage{bm}
\usepackage[hang, flushmargin]{footmisc}
\usepackage[colorlinks=true]{hyperref}
\usepackage[nameinlink]{cleveref}
\usepackage{footnotebackref}
\usepackage{url}
\usepackage{listings}
\usepackage[most]{tcolorbox}
\usepackage{inconsolata}
\usepackage[papersize={8.5in,11in}, margin=1in]{geometry}
\usepackage{float}
\usepackage{caption}
\usepackage{esint}
\usepackage{url}
\usepackage{enumitem}
\usepackage{subfig}
\usepackage{wasysym}
\newcommand{\ilc}{\texttt}
\usepackage{etoolbox}
\usepackage{algorithm}
% \usepackage{algorithmic}
\usepackage[noend]{algpseudocode}
\usepackage{tikz}
\usetikzlibrary{matrix,positioning,arrows.meta,arrows}
\patchcmd{\thebibliography}{\section*{\refname}}{}{}{}


\begin{document}



\title{\textbf{CSDS 455: Homework 1}}

\author{Shaochen (Henry) ZHONG, \ilc{sxz517}}
\date{Due and submitted on 08/26/2020 \\ CSDS 455, Dr. Connamacher}
\maketitle

\section{Problem 1}

\begin{proof}
To proof by direct proof:\newline

Let graph $G$ be a self-complementary graph with $n$ vertices, and let $G'$ be a fully connected graph based on the vertices of $G$. It is known that $G$ must have half of the edges of $G'$ to be self-complementary. Thus, we may tell $G$ has:

\begin{equation}
    \frac{{n \choose 2}}{2} = \frac{n(n-1)}{4} \nonumber
\end{equation}

number of edges.

Thus, since the number of edges must be $\in \mathbb{Z}$, we must have $n$ or $(n-1)$ being divisible by $4$. Therefore, for a self-complementary graph with $n$ vertices, we must have $n \equiv 0$ or $n \equiv 1 \mod 4$.
\end{proof}

\section{Problem 2}

\begin{proof}
To proof by contradiction:\newline
Assume there exists two disjoined paths, $P$ and $Q$, both being the longest paths in a connected graph $G$ (with length $L$), where we have vertices $p \in P$ and $q \in Q$. Since $P$ and $Q$ have no shared vertex, also since $G$ is a connected graph, there must be a path $R$ which connects vertex $p$ to $q$, where the length of $R$ must be $\geq 1$.

Make $p$ and $q$ being the mid-point of $P$ and $Q$ respectively, we may have a new path $S$ travels from the first half of $P$ (till vertex $p$), then go through $R$ (to vertex $q$), then travels thought the second half of $Q$. This path $S$ will have a length of $\ilc{len(P)} + \ilc{len(R)} + \ilc{Q}$ which is at least $\frac{1}{2}L + 1 \frac{1}{2}L$, which is $\geq L$. Thus, by contradiction, if $P$ and $Q$ are the longest paths in a connected graph, they must have a common vertex.




\section{Problem 3}

\section{Problem 4}




% \section{References}
%
% \nocite{*}
% \raggedright
% \bibliography{references.bib}
% \bibliographystyle{plain}


\end{document}