\documentclass[11pt]{article}
\usepackage{setspace}
\setstretch{1}
\usepackage{amsmath,amssymb, amsthm}
\usepackage{graphicx}
\usepackage{bm}
\usepackage[hang, flushmargin]{footmisc}
\usepackage[colorlinks=true]{hyperref}
\usepackage[nameinlink]{cleveref}
\usepackage{footnotebackref}
\usepackage{url}
\usepackage{listings}
\usepackage[most]{tcolorbox}
\usepackage{inconsolata}
\usepackage[papersize={8.5in,11in}, margin=1in]{geometry}
\usepackage{float}
\usepackage{caption}
\usepackage{esint}
\usepackage{url}
\usepackage{enumitem}
\usepackage{subfig}
\usepackage{wasysym}
\newcommand{\ilc}{\texttt}
\usepackage{etoolbox}
\usepackage{algorithm}
\usepackage{changepage}
% \usepackage{algorithmic}
\usepackage[noend]{algpseudocode}
\usepackage{tikz}
\usetikzlibrary{matrix,positioning,arrows.meta,arrows}
\patchcmd{\thebibliography}{\section*{\refname}}{}{}{}
% \PassOptionsToPackage{hyphens}{url}\usepackage{hyperref}

\providecommand{\myceil}[1]{\left \lceil #1 \right \rceil }
\providecommand{\myfloor}[1]{\left \lfloor #1 \right \rfloor }


\begin{document}



\title{\textbf{CSDS 455: Homework 9}}

\author{Shaochen (Henry) ZHONG, \ilc{sxz517}}
\date{Due and submitted on 09/23/2020 \\ Fall 2020, Dr. Connamacher}
\maketitle

\section*{Problem 1}

Suppose we have $k$ edge-disjoint $a \to b$ paths in $G$, this suggest there are at least $k$ entirely distinct ways of getting from vertex $a$ to $b$. So intuitively, we need to cut out all of these paths to ensure $a$ to be disconnected from $b$. And it is known that to cut off a path, at least 1 edge needs to be removed, thus for a graph with $k$ edge-disjoint $a \to b$ paths, we need to remove $k$ edges to make $a$ to be disconnected from $b$.

Note when selecting an edge to remove from an edge-disjoint path, such edge must be an common edge of all paths from $a \to b$ which share edge(s) with this particular edge disjoint path.


\section*{Problem 2}
\section*{Problem 3}


% \section{References}
%
% \nocite{*}
% \raggedright
% \bibliography{references.bib}
% \bibliographystyle{plain}


\end{document}