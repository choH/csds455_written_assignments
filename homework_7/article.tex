\documentclass[11pt]{article}
\usepackage{setspace}
\setstretch{1}
\usepackage{amsmath,amssymb, amsthm}
\usepackage{graphicx}
\usepackage{bm}
\usepackage[hang, flushmargin]{footmisc}
\usepackage[colorlinks=true]{hyperref}
\usepackage[nameinlink]{cleveref}
\usepackage{footnotebackref}
\usepackage{url}
\usepackage{listings}
\usepackage[most]{tcolorbox}
\usepackage{inconsolata}
\usepackage[papersize={8.5in,11in}, margin=1in]{geometry}
\usepackage{float}
\usepackage{caption}
\usepackage{esint}
\usepackage{url}
\usepackage{enumitem}
\usepackage{subfig}
\usepackage{wasysym}
\newcommand{\ilc}{\texttt}
\usepackage{etoolbox}
\usepackage{algorithm}
\usepackage{changepage}
% \usepackage{algorithmic}
\usepackage[noend]{algpseudocode}
\usepackage{tikz}
\usetikzlibrary{matrix,positioning,arrows.meta,arrows}
\patchcmd{\thebibliography}{\section*{\refname}}{}{}{}
% \PassOptionsToPackage{hyphens}{url}\usepackage{hyperref}

\providecommand{\myceil}[1]{\left \lceil #1 \right \rceil }
\providecommand{\myfloor}[1]{\left \lfloor #1 \right \rfloor }


\begin{document}



\title{\textbf{CSDS 455: Homework 7}}

\author{Shaochen (Henry) ZHONG, \ilc{sxz517}}
\date{Due and submitted on 09/16/2020 \\ Fall 2020, Dr. Connamacher}
\maketitle

\section*{Problem 1}

\textit{I consulted \url{https://math.stackexchange.com/questions/666997/} for this problem.}\newline

With the Cayley's Formula (learned on Monday's class), we know that a $K_n$ graph will have $n^{n-2}$ spanning trees. We may create a bipartite graph with partitions $T, E$ where each node $t \in T$ represent a spanning tree of $K_n$, and each $e \in E$ represent an edge in $K_n$. Then we connect these two partitions if $e \in E(t)$.

It is known that each spanning tree of $K_n$ has $n-1$ edges, and we know that there are $n^{n-2}$ nodes $\in T$. Therefore the bipartite graph will have $(n-1) \cdot n^{n-2}$ edges. \newline

Due the nature of $K_n$, every edge of it is equivalent to another; this implies the number of spanning trees containing an edge will be the same as the number of spanning trees containing any other edge. Therefore each edge is contained by $\frac{(n-1) \cdot n^{n-2}}{{n \choose 2}} = 2n^{n-3}$ (as there are ${n \choose 2}$ edges in $K_n$). We substract this number from the total number of spanning tree of $K_n$, which is same as removing an edge $e$ from $K_n$, and there will be $ n^{n-2} - 2n^{n-3} = (n-2)n^{n-3}$ spanning tree left in $K_n - e$.

\section*{Problem 2}

% \section{References}
%
% \nocite{*}
% \raggedright
% \bibliography{references.bib}
% \bibliographystyle{plain}


\end{document}