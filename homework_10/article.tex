\documentclass[11pt]{article}
\usepackage{setspace}
\setstretch{1}
\usepackage{amsmath,amssymb, amsthm}
\usepackage{graphicx}
\usepackage{bm}
\usepackage[hang, flushmargin]{footmisc}
\usepackage[colorlinks=true]{hyperref}
\usepackage[nameinlink]{cleveref}
\usepackage{footnotebackref}
\usepackage{url}
\usepackage{listings}
\usepackage[most]{tcolorbox}
\usepackage{inconsolata}
\usepackage[papersize={8.5in,11in}, margin=1in]{geometry}
\usepackage{float}
\usepackage{caption}
\usepackage{esint}
\usepackage{url}
\usepackage{enumitem}
\usepackage{subfig}
\usepackage{wasysym}
\newcommand{\ilc}{\texttt}
\usepackage{etoolbox}
\usepackage{algorithm}
\usepackage{changepage}
% \usepackage{algorithmic}
\usepackage[noend]{algpseudocode}
\usepackage{tikz}
\usetikzlibrary{matrix,positioning,arrows.meta,arrows}
\patchcmd{\thebibliography}{\section*{\refname}}{}{}{}
% \PassOptionsToPackage{hyphens}{url}\usepackage{hyperref}

\providecommand{\myceil}[1]{\left \lceil #1 \right \rceil }
\providecommand{\myfloor}[1]{\left \lfloor #1 \right \rfloor }


\begin{document}



\title{\textbf{CSDS 455: Homework 10}}

\author{Shaochen (Henry) ZHONG, \ilc{sxz517}}
\date{Due and submitted on 09/28/2020 \\ Fall 2020, Dr. Connamacher}
\maketitle

\textit{I have learned from Yige Sun for this assignment.}

\section*{Problem 1}

Let $n$ be the number of vertex, $e$ be edges, and $f$ be faces. The base case would be $n = 1$ since it is just a dot, there will be $e = 0$ and $f = 0$, thus satisfies $n - e + f = 2$.\newline

Now assume it is true for $n = k_1$, $e = k_2$, and $f = k_3$. The first case would be to add one edge without any extra vertex. An edge among two existed vertices would slice a face into two face, thus we have $e' = k_2 + 1$, $f' = k_3 +1$, and $v' = v$. Where $v' - e' + f' = 2$ still holds.

The other case would be to to add one extra vertex and one edge. In this case no extra face will be added and we have $e' = k_2 + 1$, $v' = k_1 + 1$, $f' = k_3$, where $v' - e' + f' = 2$ still holds.

Since a planar graph is connected, the only way too way we can add on it is to add an edge, or an edge with and a vertex (a path). The above too cases have shown the Eular Formular holds for both cases, thus by induction the statement is proven.

\section*{Problem 2}

By observation we know that for any planar graph with $f = 1$, such graph must be a tree and therefore must have a vertex $v$ with $d(v) < 6$.\newline

For any planar graph $G$ with $f > 1$, such graph must be $e \geq 3$ as otherwise there will not be a finite face. In those cases we may observe $\sum E(f) = 2e$ where $E(f_i)$ is the number of edges of face $f_i$, and $\sum E(f)$ represents sum of number of edges of every face in a planar graph. We know this statement is true as by counting every faces, we will eventually count each edge twice. Also, as we noticed perviously, each face will have at least 3 edges, there must also be $\sum E(f) \geq 3f$\newline

Combine the above two findings of $\sum E(f) \geq 3f$ and $\sum E(f) = 2e$, we have $2e \geq 3f $. We may apply this to the Eular Formular:

\begin{align*}
    f = 2 - n + e &\leq \frac{2}{3}e \\
    \frac{1}{3}e &\leq n - 2 \\
    e &\leq 3n - 6\\
\end{align*}

Since we know that each edge will have two vertices, there must be $d(G) = 2e$. Subsitute this finding into the above equation, we have:


\begin{align*}
e &\leq 3n - 6\\
d(G) &\leq 6n - 12 \\
\frac{d(G)}{n} &\leq 6 - \frac{12}{n}
\end{align*}

$\frac{d(G)}{n} &\leq 6 - \frac{12}{n}$ suggests the average degree per vertex in $G$ is less than $6$, which implies there will be aleast one vertex $v$ in $V(G)$ where $d(v) \leq 6$.\newline

We have proven the statement to be true for all possible cases.






\section*{Problem 3}
\section*{Problem 4}







% \section{References}
%
% \nocite{*}
% \raggedright
% \bibliography{references.bib}
% \bibliographystyle{plain}


\end{document}