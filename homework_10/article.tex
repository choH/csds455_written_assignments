\documentclass[11pt]{article}
\usepackage{setspace}
\setstretch{1}
\usepackage{amsmath,amssymb, amsthm}
\usepackage{graphicx}
\usepackage{bm}
\usepackage[hang, flushmargin]{footmisc}
\usepackage[colorlinks=true]{hyperref}
\usepackage[nameinlink]{cleveref}
\usepackage{footnotebackref}
\usepackage{url}
\usepackage{listings}
\usepackage[most]{tcolorbox}
\usepackage{inconsolata}
\usepackage[papersize={8.5in,11in}, margin=1in]{geometry}
\usepackage{float}
\usepackage{caption}
\usepackage{esint}
\usepackage{url}
\usepackage{enumitem}
\usepackage{subfig}
\usepackage{wasysym}
\newcommand{\ilc}{\texttt}
\usepackage{etoolbox}
\usepackage{algorithm}
\usepackage{changepage}
% \usepackage{algorithmic}
\usepackage[noend]{algpseudocode}
\usepackage{tikz}
\usetikzlibrary{matrix,positioning,arrows.meta,arrows}
\patchcmd{\thebibliography}{\section*{\refname}}{}{}{}
% \PassOptionsToPackage{hyphens}{url}\usepackage{hyperref}

\providecommand{\myceil}[1]{\left \lceil #1 \right \rceil }
\providecommand{\myfloor}[1]{\left \lfloor #1 \right \rfloor }


\begin{document}



\title{\textbf{CSDS 455: Homework 10}}

\author{Shaochen (Henry) ZHONG, \ilc{sxz517}}
\date{Due and submitted on 09/28/2020 \\ Fall 2020, Dr. Connamacher}
\maketitle

\textit{I have discussed with Yige Sun for this assignment.}

\section*{Problem 1}

Let $n$ be the number of vertex, $e$ be edges, and $f$ be faces. The base case would be $n = 1$ since it is just a dot, there will be $e = 0$ and $f = 0$, thus satisfies $n - e + f = 2$.\newline

Now assume it is true for $n = k_1$, $e = k_2$, and $f = k_3$. The first case would be to add one edge without any extra vertex. An edge among two existed vertices would slice a face into two face, thus we have $e' = k_2 + 1$, $f' = k_3 +1$, and $v' = v$. Where $v' - e' + f' = 2$ still holds.

The other case would be to to add one extra vertex and one edge. In this case no extra face will be added and we have $e' = k_2 + 1$, $v' = k_1 + 1$, $f' = k_3$, where $v' - e' + f' = 2$ still holds.

Since a planar graph is connected, the only way too way we can add on it is to add an edge, or an edge with and a vertex (a path). The above too cases have shown the Eular Formular holds for both cases, thus by induction the statement is proven.

\section*{Problem 2}

By observation we know that for any planar graph with $f = 1$, such graph must be a tree and therefore must have a vertex $v$ with $d(v) < 6$.\newline

For any planar graph $G$ with $f > 1$, such graph must be $e \geq 3$ as otherwise there will not be a finite face. In those cases we may observe $\sum E(f) = 2e$ where $E(f_i)$ is the number of edges of face $f_i$, and $\sum E(f)$ represents sum of number of edges of every face in a planar graph. We know this statement is true as by counting every faces, we will eventually count each edge twice. Also, as we noticed perviously, each face will have at least 3 edges, there must also be $\sum E(f) \geq 3f$\newline

Combine the above two findings of $\sum E(f) \geq 3f$ and $\sum E(f) = 2e$, we have $2e \geq 3f $. We may apply this to the Eular Formular:

\begin{align*}
    f = 2 - n + e &\leq \frac{2}{3}e \\
    \frac{1}{3}e &\leq n - 2 \\
    e &\leq 3n - 6\\
\end{align*}

Since we know that each edge will have two vertices, there must be $d(G) = 2e$. Subsitute this finding into the above equation, we have:


\begin{align*}
e &\leq 3n - 6\\
d(G) &\leq 6n - 12 \\
\frac{d(G)}{n} &\leq 6 - \frac{12}{n}
\end{align*}

$\frac{d(G)}{n} \leq 6 - \frac{12}{n}$ suggests the average degree per vertex in $G$ is less than $6$, which implies there will be aleast one vertex $v$ in $V(G)$ where $d(v) \leq 6$.\newline

We have proven the statement to be true for all possible cases.




\section*{Problem 3}

\textit{I have consulted \url{https://math.stackexchange.com/questions/764566} and \url{http://www.su18.eecs70.org/static/slides/lec-5-handout.pdf} for this problem.}
\newline

Although being an \textit{iff} question, we will only need to prove one direction as the dual graph of $G$ is $G*$ and so does vice versa.\newline

Let $T$ be the spanning tree of $G$, and let $T^*$ be a subgraph of $G^*$ consisting edges of $G^*$ that are corresponding to the edges of $G$ not in $T$.

Known that $|V(T)| = |V(G)| - 1$, thus there must be $|E(T^*)| = |E(G)| - (|V(G)| - 1)$. By Euler's formula there must be $2 - |V(G)| + |E(G)|$ faces in $G$; since we know that a face in $G$ corresponds to a vertex in $G^*$, we have:

\begin{align*}
    |V(G^*)| &= 2 - |V(G)| + |E(G)| \\
    &= |E(G)|- |V(G)| + 2 \\
    |E(T^*)| &= |E(G)| - (|V(G)| - 1) \\
    &= |E(G)| - |V(G)| + 1 \\
    &= |V(G^*)| - 1
\end{align*}

Now we have shown that $|E(T^*)| = |V(G^*)| - 1$. To show that $T^*$ is a spanning tree of $G^*$, we need to show that $T^*$ is an acyclic, connected subgraph of $G*$ that consists every vertices of $G*$.

\leavevmode\newline


    \begin{adjustwidth}{1cm}{}

    \begin{proof}
    \textbf{Cycle-Cut Duality: An edge cut in $G^*$ is a cycle in $G$}.\newline

    A vertex in $G^*$ corresponds to a face in $G$. This suggests if we can perform an edge cut to disjoint one or couple vertices $\in V(G^*)$ from $G^*$, the removed edges in $G*$ will correspond to some faces enclosing edges in $G$. Since we know that a face is always enclosed by a cycle of edges, a cut in $G^*$ is a cycle in $G$.

    \textbf{Promoting this conclusion known that $G$ is the dual of $G*$ and vise versa, a cycle in $G$ also corresponds a cut in $G*$.}


    \end{proof}

    \end{adjustwidth}

We first show that $T^*$ is acyclic. This can be conclude by observating the fact that $T$ is connected, which implies $V(G) - T$ is have no edge cut of $G$. Thus, by the lemma, there no cycle in $T^*$ and $T^*$ is therefore acyclic.

We also know that $T^*$, being acyclic subgraph of $G^*$ with $|V(G^*)| - 1$ edges, must also be connected. Since after the first vertex of $T^*$, every edge of it will bring in an extra vertex, and there are only $|V(G^*)|$ vertices in $G^*$ -- which will all be included in $T^*$.\newline

Knowing that $T^*$ is an connected, acyclic subgraph of $G^*$ with $|V(G^*)|$ verticies. We have proven that $T^*$ is a spanning tree of $G^*$.


% We will do it by induction on number of faces. For $f = 1$, $G$ must be a tree where $G^*$ must be a node (since there is only one infinite face). Therefore it is trivially true that the duals of remaining edges of $G$ form a spanning tree of $G^*$.
%
% Assume we assume this is also the case for a $G$ with $k$ faces,
%






\section*{Problem 4}


Assume a minimum 2-connected graph $G$ with $f = 2$ (which is a triangle), we know the dual of $G$, $G^*$ will also be 2-connected -- as the $G^*$ in this case only has two vertices each with degree of $3$. So obviously removing one vertex won't be able to make this $G^*$ disconnected. So the base case is trivially true.\newline

Proceed to do induction on number of faces, assume that it is also true for a 2-connected graph $G_k$ with $f=k$. There are two ways we may have a 2-connected $G_{k+1}$ with $f = k+1$.\newline

First we simply add an edge between the existing vertices of $G_k$, this will split a face $x$ in $G_k$ into $x_1$ and $x_2$ in $G_{k+1}$. This $G_{k+1}$ must be 2-connected since $G_k$ is 2-connected, and there is no way that adding an edge (or edges) may result a decrease in vertex-connectivity. We denotes the vertex representing face $x$ of $G_k$ in $G^*_k$ as $v_x$, and simliar for $v_{x_1}, v_{x_2}$.

The $G^*_{k+1}$ in this case will be $G^*_k - \{v_x\}$ but with new edges to and between $v_{x_1}$ and $v_{x_2}$. Known that $G^*_k$ is 2-connected, which means $G^*_k - \{v_x\}$ must still be connected. So as long as $v_{x_1}$ and/or $v_{x_2}$ won't be disconnected from the rest of the $G^*_{k+1}$ with one vertex removed, we have shown $G^*_{k+1}$ to be 2-connected.


We know this is true as each of $x_1$ and $x_2$ must be enclosed by 3 edges for minimum to be a face. So there must be 3 edges coming out of $v_{x_1}$ and 3 edges coming out of $v_{x_2}$. W.L.O.G. we know that 1 edge of $v_{x_1}$ will be $v_{x_1}v_{x_2}$, so even assume the other 2 edges of $v_{x_1}$ are connected to the same vertex other than $v_{x_2}$, this $v_{x_1}$ will not be 1-vertex-disconnectable to the rest of the $G^*_{k+1}$. This inference also applies to $v_{x_2}$. So with $x_1$ and $x_2$ being 2-connected, $G^*_{k+1}$ will be 2-connected.\newline


The second case will be to connect (or ``merge'') a 1-face component to $G_k$ to make $G_{k+1}$ with $k+1$ faces. To keep $G_{k+1}$ 2-connected, this new component with a face denotes $x_1$ must share at least 2 vertices with $G_k$ (a.k.a. adding a path between 2 vertices of  $G_k$); as otherwise if only 1 vertex is shared, we may simply remove this shared vertex and $G_{k+1}$ will be disconnected.

With the two shared vertices, this new face $x_1$ will be ``neighbered'' with an existed face in $G_k$ (call that face $y$), and this new face $x_1$ will be inside another face $x$ in $G_k$ (and this face $x$ is now splited into faces $x_1$ and $x_2$ in $G_{k+1}$).

Now we have converted this case to be same as the first case: one face in $G_k$ splitted into 2 faces in $G_{k+1}$. And we therefore know that this $G^*_{k+1}$ will still be 2-connected by the same conclusion.\newline

Since we have shown that for a 2-connected planar graph $G$ with any number of faces, its dual graph $G^*$ will also be 2-connected under all possible cases, we have proven the statement by induction.





% \section{References}
%
% \nocite{*}
% \raggedright
% \bibliography{references.bib}
% \bibliographystyle{plain}


\end{document}