\documentclass[11pt]{article}
\usepackage{setspace}
\setstretch{1}
\usepackage{amsmath,amssymb, amsthm}
\usepackage{graphicx}
\usepackage{bm}
\usepackage[hang, flushmargin]{footmisc}
\usepackage[colorlinks=true]{hyperref}
\usepackage[nameinlink]{cleveref}
\usepackage{footnotebackref}
\usepackage{url}
\usepackage{listings}
\usepackage[most]{tcolorbox}
\usepackage{inconsolata}
\usepackage[papersize={8.5in,11in}, margin=1in]{geometry}
\usepackage{float}
\usepackage{caption}
\usepackage{esint}
\usepackage{url}
\usepackage{enumitem}
\usepackage{subfig}
\usepackage{wasysym}
\newcommand{\ilc}{\texttt}
\usepackage{etoolbox}
\usepackage{algorithm}
\usepackage{changepage}
% \usepackage{algorithmic}
\usepackage[noend]{algpseudocode}
\usepackage{tikz}
\usetikzlibrary{matrix,positioning,arrows.meta,arrows}
\patchcmd{\thebibliography}{\section*{\refname}}{}{}{}
% \PassOptionsToPackage{hyphens}{url}\usepackage{hyperref}

\providecommand{\myceil}[1]{\left \lceil #1 \right \rceil }
\providecommand{\myfloor}[1]{\left \lfloor #1 \right \rfloor }


\begin{document}



\title{\textbf{CSDS 455: Homework 17}}

\author{Shaochen (Henry) ZHONG, \ilc{sxz517}}
\date{Due and submitted on 10/21/2020 \\ Fall 2020, Dr. Connamacher}
\maketitle

Hi Kyle, I got one midterm comming so didn't invest as much of time on this as I used to. Sorry if the proofs is kind of sketchy.

\section*{Problem 1}

\textit{I have consulted \url{https://www.ti.inf.ethz.ch/ew/lehre/GA10/lec-nfz-new-nopause.pdf} for this problem.}\newline


I think it has something to do with Tutte's flow conjectures where every bridgeless graph has a 5-NZF. We already know that the sum of flow along any edge-cut of the graph is 0. So every bridgeless graph must admit a $k$-NZF. Then we push the $k$ with Tutte's flow conjectures till $k=5$ as demonstrated in the referenced source.\newline

\noindent The other way to think about it might be the fact that if $G$ is $4$-NZF, then the flow of an edge $e$ of $G$ is at most $3$ and never $0$. Assume $e$ is an edge of verticies $uv$ in $G$. Since $G-e$ is bridgeless, this suggests $G$ is at least 3-connected. So maybe by distributing the $f(e) = 3$ to each edge, we promote some edge $e'$ to at most $f(e') = 4$ and therefore become $5$-NZF. But I haven't fully rationalized how to do this flow re-distribution yet.


\section*{Problem 2}

\textit{I have consulted \url{http://www.people.vcu.edu/~dcranston/slides/nowhere-zero-talk.pdf} and Yuhui Zhang for this problem.}\newline

Let $G = G_1 \cup G_2$, say we have flow $f_1, f_2$ on $G_1$ and $G_2$ respectively. We extend $f_1$ to $\hat{f_1}$ by assigning weight $0$ to edges $\in E(G) - E(G_1)$; and likewise, extend $f_2$ to $\hat{f_2}$ by by assigning weight $0$ to edges $\in E(G) - E(G_2)$.

Let $f = \hat{f_1} + k_1 \hat{f_2}$ which is a non-zero flow as $f_1$ and $f_2$ are non-zero. We know that $|f(e) \leq (k_1 - 1) + k_1(k_2 -1) = k_1 k_2 -1$. The statement is therefore proven.







\end{document}