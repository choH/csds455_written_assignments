\documentclass[11pt]{article}
\usepackage{setspace}
\setstretch{1}
\usepackage{amsmath,amssymb, amsthm}
\usepackage{graphicx}
\usepackage{bm}
\usepackage[hang, flushmargin]{footmisc}
\usepackage[colorlinks=true]{hyperref}
\usepackage[nameinlink]{cleveref}
\usepackage{footnotebackref}
\usepackage{url}
\usepackage{listings}
\usepackage[most]{tcolorbox}
\usepackage{inconsolata}
\usepackage[papersize={8.5in,11in}, margin=1in]{geometry}
\usepackage{float}
\usepackage{caption}
\usepackage{esint}
\usepackage{url}
\usepackage{enumitem}
\usepackage{subfig}
\usepackage{wasysym}
\newcommand{\ilc}{\texttt}
\usepackage{etoolbox}
\usepackage{algorithm}
\usepackage{changepage}
% \usepackage{algorithmic}
\usepackage[noend]{algpseudocode}
\usepackage{tikz}
\usetikzlibrary{matrix,positioning,arrows.meta,arrows}
\patchcmd{\thebibliography}{\section*{\refname}}{}{}{}
% \PassOptionsToPackage{hyphens}{url}\usepackage{hyperref}

\providecommand{\myceil}[1]{\left \lceil #1 \right \rceil }
\providecommand{\myfloor}[1]{\left \lfloor #1 \right \rfloor }


\begin{document}



\title{\textbf{CSDS 455: Homework 24}}

\author{Shaochen (Henry) ZHONG, \ilc{sxz517}}
\date{Due and submitted on 11/16/2020 \\ Fall 2020, Dr. Connamacher}
\maketitle

\textit{I have consulted Yige Sun for this assignment.}\newline

\section*{Problem 1}

\textit{Using the idea of "ribbons", modify CLIQUE example to determine the existance of 6-cliques instead of 4-cliques, and modify the TRIANGLE example to count the number of 5-cliques instead of triangles.}\newline

\noindent \ilc{6-CLIQUE} and \ilc{5-CLIQUE} with utilization of ribbons can be expressed as:


\begin{equation*}
    \ilc{6-CLIQUE}(A_R, B_R) = \frac{1}{2^{15}} \sum\limit_{\text{R: graph on} \ V(A_r \cup B_r)} \Pi\limit_{\{l_1, l_2\} \in E(R)} \chi \{l_1, l_2\} \\
\end{equation*}




\begin{equation*}
    \ilc{5-CLIQUE}(A_R, B_R) = \frac{1}{2^{10}} \sum\limit_{w_1 \in [n] \char`\\ V(A_r \cup B_r)} \ \ \cdot \sum\limit_{\text{R: graph on} \ V(A_r \cup B_r \cup C_R)} \Pi\limit_{\{l_1, l_2\} \in E(R)} \chi \{l_1, l_2\}
\end{equation*}



\section*{Problem 2}
\textit{Is there a parity issue with this technique?  Can we count the number of 6-cliques and determine the existance of 5-cliques?  Show how you can adjust the algorithms for these problems or show why a straightforward modification will not work.}\newline

\noindent I don't quite get this question, if we can count the number of 6-cliques we can surely determine the exsitance of 5-cliques, as any 6-clique contains a 5-clique. But in the case of no 6-clique, there is still the possibility of having 5-clique. In such case we may remove a vertex $x$ from one side of the ribben and run the \ilc{n-CLIQUE} algorithm to count if there's a $\ilc{(n-1)-CLIQUE}$.


\section*{Problem 3}
\textit{Given your solutions above, define what the "ribbon" and "shape" would be for the solutions.}\newline

\begin{itemize}
    \item Ribbon in \ilc{6-CLIQUE} in \textbf{Problem 1 & 2} is two sets of verticies of equal size (3).
    \item Shape in \ilc{5-CLIQUE} in \textbf{Problem 1} is two sets of verticies of equal size (2) and another vertex from $[n]$.
\end{itemize}


\section*{Problem 4}

\textit{Describe (general terms) the purpose of the "constraint graph" in section 3.  What is it supposed to represent?}\newline

\noindent I haven't got this far yet but from a superficial scan it seems it a multi-graph representation of multiple graph matrices ``gluting'' together. Specifically it makes edge if there is an edge between two verticies from different partition of the ribben.


\end{document}