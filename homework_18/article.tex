\documentclass[11pt]{article}
\usepackage{setspace}
\setstretch{1}
\usepackage{amsmath,amssymb, amsthm}
\usepackage{graphicx}
\usepackage{bm}
\usepackage[hang, flushmargin]{footmisc}
\usepackage[colorlinks=true]{hyperref}
\usepackage[nameinlink]{cleveref}
\usepackage{footnotebackref}
\usepackage{url}
\usepackage{listings}
\usepackage[most]{tcolorbox}
\usepackage{inconsolata}
\usepackage[papersize={8.5in,11in}, margin=1in]{geometry}
\usepackage{float}
\usepackage{caption}
\usepackage{esint}
\usepackage{url}
\usepackage{enumitem}
\usepackage{subfig}
\usepackage{wasysym}
\newcommand{\ilc}{\texttt}
\usepackage{etoolbox}
\usepackage{algorithm}
\usepackage{changepage}
% \usepackage{algorithmic}
\usepackage[noend]{algpseudocode}
\usepackage{tikz}
\usetikzlibrary{matrix,positioning,arrows.meta,arrows}
\patchcmd{\thebibliography}{\section*{\refname}}{}{}{}
% \PassOptionsToPackage{hyphens}{url}\usepackage{hyperref}

\providecommand{\myceil}[1]{\left \lceil #1 \right \rceil }
\providecommand{\myfloor}[1]{\left \lfloor #1 \right \rfloor }


\begin{document}



\title{\textbf{CSDS 455: Homework 18}}

\author{Shaochen (Henry) ZHONG, \ilc{sxz517}}
\date{Due and submitted on 10/26/2020 \\ Fall 2020, Dr. Connamacher}
\maketitle


\section*{Problem 1}

Known that every planer graph can be drawn in a plan graph format, we convert our planer $G$ to plane $G$. Now for edge contraction: if the contracted edge $e$ is a part of a chordal (3-edge) circle in $G$, then after the contraction every chordal circle including this $e$ will be ``collapsed'' to a line, and such collapse will not creating any non-chordal structure, thus the graph is still chordal. If the contracted edge $e$ is not part of a chordal circle, then a contraction of $e$ won't affect the chordal property of the graph.

Since the graph is still chordal regardless which edge $e$ is contracted, the statement is therefore proven

\section*{Problem 2}
\section*{Problem 3}
\section*{Problem 4}








\end{document}